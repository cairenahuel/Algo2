\documentclass[10pt,a4paper]{article}
\input{AEDmacros}
\input{nahuelMacros}
\usepackage{subfiles}
\usepackage{caratula} % Version modificada para usar las macros de algo1 de ~> https://github.com/bcardiff/dc-tex
\usepackage{graphicx} % Required for inserting images
\usepackage{amsfonts} % Required for inserting images
\usepackage{amsmath}
\usepackage{color}


\titulo{Trabajo Practico N°1}
\subtitulo{Fondo Monetario Cómun}

\fecha{\today}

\materia{Algoritmos y Estructuras de Datos}
\grupo{Grupo "Nominadores"}

\integrante{Caire, Nahuel A.}{1140/23}{cairenahuel@gmail.com}
\integrante{Motta, Marino J.}{1372/23}{marijmotta@gmail.com}
\integrante{Rabey, Nahuel A.}{1394/23}{nahuelrabeywork@gmail.com}
\integrante{Sola, Santiago}{42/23}{santiagoms.ss@gmail.com}
% Pongan cuantos integrantes quieran

% Declaramos donde van a estar las figuras
% No es obligatorio, pero suele ser comodo
\graphicspath{{../static/}}

\begin{document}
\maketitle
\textbf{\Large{Auxiliares y predicados \small{utilizados en mas de un procedimiento.}}}\\
\subfile{auxiliares}
\section*{Ejercicio 1.1:}
\subfile{ejercicio1.tex}
\section*{Ejercicio 1.2:}
\subfile{ejercicio2Alter.tex}
\pagebreak
\section*{Ejercicio 1.3:}
\subfile{ejercicio3.tex}
\section*{Ejercicio 1.4:}
\subfile{ejercicio4.tex}
\section*{Ejercicio 1.5:}
\subfile{ejercicio5.tex}
\section*{Ejercicio 2}
\subfile{ejercicioIIpreludio.tex}
\subfile{ejercicioII.tex}
\section*{A los profesores}

Claramente este trabajo práctico está incompleto. En la demostración, no cumplí ninguno de los objetivos planteados al principio. Ni que la poscondición implique el asegura, ni que la precondición implique el requiere. Sin mencionar los dudos métodos deductivos para justificar la validez de los argumentos.

Lo propio vale para el ejercicio 1, aunque mucho mejor preparado, entendemos que hay algunas libertades sintácticas (por ser generosos) que quizás no están permitidas.

En general, pudimos haber hecho un mucho mejor trabajo, pero nos confiamos en el tiempo que disponíamos para completarlos.

Desde ya, muchas gracias.
\end{document}

\documentclass[document.tex]{subfiles}
\begin{document}
\begin{proc}{redistribucionDeLosFrutos}{\In recursos : \TLista{\float}, \In cooperan: \TLista{\bool}}{\TLista{\float}}
\requiere{|recursos|=|cooperan|}
%\requiere{|recursos|>0} no parece hacer falta pero ¿esta de mas ponerla?
\requiere{\paraTodo[unalinea]{n}{\ent}{todosPositivos(n,recursos)}}
\asegura{|res|=|recursos|}
\asegura{\paraTodo[unalinea]{n}{\ent}{enRango(n, res) \implicaLuego res[n]=(\IfThenElse{cooperan[n]}{ufc}{recursos[n]+ufc})}}
%\aux{unidadFondoComun}{\In recursos: \TLista{\float}, \In cooperan: \TLista{\bool}}{\float}{\\(\sum_{i=0}^{|recursos|-1} \IfThenElse{cooperan[i]}{recursos[i]}{0})/|recursos|}
%\texttt{\hspace{0.5cm} donde ufc=unidadFondoComun(recursos, cooperan)}
\begin{verbatim}
    donde ufc=unidadFondoComun(recursos, cooperan)}
\end{verbatim}
\end{proc}
\end{document}
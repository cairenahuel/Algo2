\documentclass[document.tex]{subfiles}
\begin{document}
%\begin{center}
Vemos que $pc\implies I$ vale. Por hipotesis sabemos que $i=0$ y $res=recursos$ son verdaderos\\

Si $i=0 \implies 0\leq i < |eventos|$. Además $S_e(i)=subset(eventos,0,0)=\emptyset$, por lo tanto $\#(S_e(i),T)=0$ y $\#(S_e(i),F)=0$

A partir de esto podemos ver que el valor de res, coincide con la hipótesis

\begin{equation}
res=r\cdot(a_c\cdot p_c)^0\cdot (a_s\cdot p_s)^0=recursos
\end{equation}

Por lo tanto, vale la siguiente implicación

\begin{equation}
(i=0 \yLuego res=r)\implies (0\leq i < |eventos| \yLuego res=r\cdot(a_c\cdot p_c)^{\#(S_c(i),T)}\cdot(a_s\cdot p_s)^{\#(S_s(i),F)})
\end{equation}

Que es lo que queríamos probar.

Vemos que $I\land ¬B \implies Q_c$. Por hipótesis, el siguiente predicado es verdadero

\begin{equation}
(\forall i \in \ent): 0 \leq i \leq |eventos| \yLuego res=r\cdot (a_c\cdot p_c)^{\#S_c(i),T}\cdot (a_s\cdot p_s)^{\#S_s(i),F}
\land i=|eventos|
\end{equation}

Observemos que $i=|eventos|$ es la negación de B, pues es el mayor valor de $i$ posible luego del ciclo, que a su vez no sea la guarda.

Luego $S_c(i)=subset(eventos, 0, |eventos|)=eventos$

Con esto, podemos calcular la cantidad de veces que aparece cara y seca 

\begin{equation}
\begin{split}
    \#(S_c(i),T)&=\#(eventos,T)\land \#(S_c(i),F)=\#(eventos,F)
\end{split}
\end{equation}

% \begin{equation}
% \implies \#(S_c(i),T)=\#(eventos,T)\yLuego \#(S_c(i),F)=\#(eventos,F)\\
% \end{equation}
Con esta información, podemos calcular el valor de res
\begin{equation}
res=r\cdot (a_c\cdot p_c)^{\#(eventos,T)}\cdot (a_s\cdot p_s)^{\#(eventos,F)}
\end{equation}

Por lo tanto $(I\land¬B)\implies Q_c$ vale, que es lo que queríamos mostrar

%\end{center}
\end{document}
\documentclass[../document.tex]{subfiles}
\begin{document}

\subsection*{Predicado 3}

Queremos ver que $I\land \neg B \implies Q_c$. Por hipótesis, lo siguiente es verdadero

\begin{enumerate}
    \item $i \geq |eventos|$
    \item $0\leq i \leq |eventos|$
    \item $res=r\cdot (a_c\cdot p_c)^{\#S_c(i),T}\cdot (a_s\cdot p_s)^{\#S_s(i),F}$
\end{enumerate}

Por (1) y (2) sé que $i=|eventos|$

Luego $S_e(i)=subset(eventos, 0, |eventos|)=eventos$

Con esto, podemos calcular la cantidad de veces que aparece cara y seca 

\begin{equation}
\begin{split}
    \#(S_c(i),T)&=\#(eventos,T)\land \#(S_c(i),F)=\#(eventos,F)
\end{split}
\end{equation}

% \begin{equation}
% \implies \#(S_c(i),T)=\#(eventos,T)\yLuego \#(S_c(i),F)=\#(eventos,F)\\
% \end{equation}
Con esta información, podemos calcular el valor de res
\begin{equation}
res=r\cdot (a_c\cdot p_c)^{\#(eventos,T)}\cdot (a_s\cdot p_s)^{\#(eventos,F)}
\end{equation}

Por lo tanto $(I\land \neg B)\implies Q_c$ vale, que es lo que queríamos mostrar

\end{document}
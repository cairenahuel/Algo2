\documentclass[../document.tex]{subfiles}
\usepackage[dvipsnames]{xcolor}
\begin{document}

\subsection*{Corolario}

Hemos demostrado las cinco propiedades que nos garantizan que $\{Pc\}C\{Qc\}$ es una tripla válida. 
Lo último que necesitamos es estudiar el comienzo del código.

Sea $S=S1;S2$, dónde $S1= res:=recursos$ y $S2= i:=0$, queremos ver que $wp(S,Pc) \implies P$

Sabemos que $wp(S,Pc)=wp(S1,wp(S2,Pc))$

Estudiemos $wp(S2,Pc)$
\begin{align}
   wp(S2,Pc) &\equiv wp(i:=0;Pc)\\
    & \equiv Pc_{0}^{i} \\
    & \equiv0 = 0 \land_{L} res = recursos \land_{L} apuesta_{c}+apuesta_{c}=1 \land_{L} pago_{c},pago_{s},apuesta_{c},apuestsa{s},recurso>0\\
    & \equiv True \land_{L} res = recursos \land_{L} apuesta_{c}+apuesta_{c}=1 \land_{L} pago_{c},pago_{s},apuesta_{c},apuestsa{s},recurso>0\\
    & \equiv res = recursos \land_{L} apuesta_{c}+apuesta_{c}=1 \land_{L} pago_{c},pago_{s},apuesta_{c},apuestsa{s},recurso>0
\end{align}

Estudiemos $wp(S1,wp(S2,Pc))$. Llamaremos a $Pf=wp(S2;Pc)$
\begin{align}
   wp(S1,Pf) &\equiv wp(res=recursos;Pf)\\
    & \equiv Pf_{recursos}^{res} \\
    & \equiv recursos = recursos \land_{L} apuesta_{c}+apuesta_{c}=1 \land_{L} pago_{c},pago_{s},apuesta_{c},apuestsa{s},recurso>0\\
    & \equiv True \land_{L} apuesta_{c}+apuesta_{c}=1 \land_{L} pago_{c},pago_{s},apuesta_{c},apuestsa{s},recurso>0\\
    & \equiv apuesta_{c}+apuesta_{c}=1 \land_{L} pago_{c},pago_{s},apuesta_{c},apuestsa{s},recurso>0
\end{align}

Efectivamente, $wp(S,Pc) \implies {P}$. Por lo tanto, es una implementación válida de nuestra especificación.

\end{document}
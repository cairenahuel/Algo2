\documentclass[../document.tex]{subfiles}
\begin{document}

Tenemos el siguiente código, el cual llamaremos $C$
\begin{verbatim}
    res := recursos
    i := 0
    while (i < |eventos|) do
        if (eventos[i]) then
            res := res * apuesta.c * pago.c
        else
            res := res * apuesta.s * pago.s
        endif
        i := i+1
    endwhile
\end{verbatim}

Para demostrar que este código cumple con el requerimiento pedido, se deben comprobar dos cosas:

\begin{enumerate}
    \item la poscondición $Q$, de la tripla de Hoare $\{P\}C\{Q\}$, debe implicar el predicado que asegura la especificación.
    \item el predicado que requiere la especificación debe implicar a la precondición {P}.
\end{enumerate}

Para verificar esto, en primer lugar, debemos estudiar el ciclo que se encuentra dentro del código.

$\{Q_c: i = 0 \land_{L} res = recursos \land_{L} apuesta_{c}+apuesta_{c}=1 \land_{L} pago_{c},pago_{s},apuesta_{c},apuestsa{s},recurso>0\}$
\begin{verbatim}
    while (i < |eventos|) do
        if (eventos[i]) then
            res := res * apuesta.c * pago.c
        else
            res := res * apuesta.s * pago.s
        endif
        i := i+1
    endwhile
\end{verbatim}
$\{P_c: i = |eventos| \land_{L} res = recursos \cdot (apuestas.c\cdot pagos.c)^{\#T}(apuestas.s\cdot pagos.c)^{\#F}\}$

donde $\#T = \#(eventos, True)$ y $\#F = \#(eventos, F)$. Proponemos, a partir de esto, el siguiente invariante

$$I \equiv (\forall i \in \mathbb{Z})(0\leq i \leq |eventos| \land_L res = recursos \cdot (apuestas.c\cdot pagos.c)^{\#T(i)}(apuestas.s\cdot pagos.c)^{\#F(i)}$$

donde $\#T(i) = \#(eventos[0..i], True)$ y $\#F(i) = \#(eventos[0..i], F)$

Además, nos quedan la siguiente guarda y función variante

$$B \equiv i < |eventos|$$
$$f_v = |eventos|-1$$

La poscondición del ciclo, $Q_c$ es el $Q$ que estamos buscando. Por lo tanto, $Q_c$ debe implicar el predicado del asegura. La precondición del ciclo nos permitirá, luego, encontrar la precondición más débil del código, y así ver si la precondición implica ésta última.

Demostremos que $\{P_c\}S\{Q_c\}$ es una tripla de Hoare válida.

Digamos que \verb+C=C1;C2+, dónde \verb+C1+ es
\begin{verbatim}
    if (eventos[i]) then
        res := res * apuesta.c * pago.c
    else
        res := res * apuesta.s * pago.s
    endif
\end{verbatim}

y \verb+C2+ es
\begin{verbatim}
    i := i+1
\end{verbatim}

En principio, \textbf{sí el ciclo finaliza}, para ser válida deben ser ciertos los siguientes predicados
\begin{enumerate}
    \item $P_c \to I$
    \item $\{I \land B\}C\{I\}$
    \item $\{I \land \neg B\} \to Q_c$
\end{enumerate}

Veamos que $\{I \land B\}C\{I\}$ sea una tripla de Hoare válida. Busco que $\verb+wp(C;I)+ \to \{I \land B\}$.

Por axioma, sabemos que \verb+wp(C;I)=wp(C1;wp(C2;I))+. Estudiempos entonces \verb|wp(C2;I)|

\begin{equation} \label{eq1}
\begin{split}
wp(C2;I) & \equiv wp(i := i+1;I) \\
         & \equiv I^{i}_{i+1} \\
         & \equiv (\forall i \in \mathbb{Z})\\
         &(0\leq i+1 \leq |eventos| \land_L\\ 
         &recursos \cdot (apuestas.c\cdot pagos.c)^{\#T(i+1)}(apuestas.s\cdot pagos.c)^{\#F(i+1)}
\end{split}
\end{equation}
Llamaremos \verb|P2=wp(C2;I)|

Ahora, debemos ver que pasa con \verb+wp(C;I)=wp(C1;P2)+, donde $P2$ fue calculado en \ref{eq1}

\begin{equation} \label{eq2}
\begin{split}
wp(C1;P2) & \equiv wp(\text{if B then C1A else C1B endif};P2) \\
          & \equiv (B \land wp(C1A, P2) \lor (\neg B \land wp(C1B, P2)) \\
\end{split}
\end{equation}
Donde
\begin{itemize}
    \item \verb|B = eventos[i]|
    \item \verb|C1A = res := res * apuesta.c * pago.c|
    \item \verb|C1B = res := res * apuesta.s * pago.s|
\end{itemize}

\begin{equation} \label{eq2.a}
\begin{split}
wp(C1A;P2) & \equiv wp(res := res * apuesta.c * pago.c; P2) \\
           & \equiv P2^{res}_{res * apuesta.c * pago.c} \\
\end{split}
\end{equation}
Diremos que $P1A = wp(C1A;P2) = P2^{res}_{res * apuesta.c * pago.c}$

\begin{equation} \label{eq2.a}
\begin{split}
P1A & \equiv 0\leq i+1 \leq |eventos| \land_L \\& res = recursos \cdot (apuestas.c\cdot pagos.c)^{\#T(i+1)}(apuestas.s\cdot pagos.c)^{\#F(i+1)}\\
P1A & \equiv 0\leq i+1 \leq |eventos| \land_L \\& res * apuesta.c * pago.c = recursos \cdot (apuestas.c\cdot pagos.c)^{\#T(i+1)}(apuestas.s\cdot pagos.c)^{\#F(i+1)}\\
P1A & \equiv 0\leq i+1 \leq |eventos| \land_L \\& res = recursos \cdot (apuestas.c\cdot pagos.c)^{\#T(i+1)-1}(apuestas.s\cdot pagos.c)^{\#F(i+1)}\\
\end{split}
\end{equation}

Observemos que $\#T(i+1)-1$ implica no contar un valor $True$ de los que se encuentran en eventos, lo cual tiene todo el sentido del mundo, porque estamos buscando una precondición a un cálculo que aún no se ha hecho.

Además, $0\leq i+1 \leq |eventos| \Longrightarrow -1\leq i \leq |eventos|-1$. Sabemos que el ciclo empieza con $i=0$, entonces $0\leq i \leq |eventos|-1$.

Entonces, podemos observar que $\#T(i+1)-1$ no se rompe, pues cómo $i \leq |eventos|-1$ el valor máximo que obtendrá es $\#T(|eventos|-1+1)-1$ y esto es igual a $\#T-1$.

\begin{equation} \label{P1A}
    P1A \equiv 0\leq i \leq |eventos|-1 \land_L \\ res = recursos \cdot (apuestas.c\cdot pagos.c)^{\#T(i+1)-1}(apuestas.s\cdot pagos.c)^{\#F(i+1)}\\
\end{equation}

\begin{equation} \label{eq2.a}
\begin{split}
wp(C1B;P2) & \equiv wp(res := res * apuesta.s * pago.s; P2) \\
           & \equiv P2^{res}_{res * apuesta.s * pago.s} \\
\end{split}
\end{equation}

Diremos que $P1B = wp(C1B;P2) = P2^{res}_{res * apuesta.s * pago.s}$

\begin{equation} \label{eq2.b}
\begin{split}
P1B & \equiv 0\leq i+1 \leq |eventos| \land_L \\& res = recursos \cdot (apuestas.c\cdot pagos.c)^{\#T(i+1)}(apuestas.s\cdot pagos.c)^{\#F(i+1)}\\
P1B & \equiv 0\leq i+1 \leq |eventos| \land_L \\& res * apuesta.s * pago.s = recursos \cdot (apuestas.c\cdot pagos.c)^{\#T(i+1)}(apuestas.s\cdot pagos.c)^{\#F(i+1)}\\
P1B & \equiv 0\leq i+1 \leq |eventos| \land_L \\& res = recursos \cdot (apuestas.c\cdot pagos.c)^{\#T(i+1)}(apuestas.s\cdot pagos.c)^{\#F(i+1)-1}\\
\end{split}
\end{equation}

Observemos que con $\#F(i+1)-1$ ocurre lo mismo que en \ref{eq2.a}

\begin{equation} \label{P1A}
    P1B \equiv 0\leq i \leq |eventos|-1 \land_L \\ res = recursos \cdot (apuestas.c\cdot pagos.c)^{\#T(i+1)}(apuestas.s\cdot pagos.c)^{\#F(i+1)-1}\\
\end{equation}

Finalmente, podemos sitetizar todo lo anterior en

\begin{equation} \label{final}
\begin{split}
    wp(C1;P2) &\equiv (B \land wp(C1A, P2) \lor (\neg B \land wp(C1B, P2))\\
    wp(C1;P2) &\equiv ((B \land P1A) \lor (\neg B \land P1B))\\
\end{split}
\end{equation}

Bien, con esto en mente, podemos ver un elemento que se repite \\$0\leq i \leq |eventos|-1$. Observemos que podemos acotarlo, de la siguiente manera\\
$0\leq i \leq |eventos|$. Al modificar el rango, también debemos adecuar $\#T(i+1)=\#T(i)$ y $\#F(i+1)=\#F(i)$

Además, tenemos $P1A \lor P1B$ que representa la unión de dos conjuntos. El primero, P1A dónde eventos[i] es cara y P1B dónde eventos[i] es seca. En ambos casos se cuenta la cantidad de caras y secas que existen en eventos. Por lo tanto

\begin{equation}
\begin{split}
P1A \lor P1B \equiv \\
    &(0\leq i \leq |eventos|-1 )\land_L\\
    &(res = recursos \cdot (apuestas.c\cdot pagos.c)^{\#T(i+1)-1}(apuestas.s\cdot pagos.c)^{\#F(i+1)} \lor \\
    & res = recursos \cdot (apuestas.c\cdot pagos.c)^{\#T(i+1)}(apuestas.s\cdot pagos.c)^{\#F(i+1)-1})\\
P1A \lor P1B \equiv \\
    &(0\leq i \leq |eventos|)\land_L\\
    &(res = recursos \cdot (apuestas.c\cdot pagos.c)^{\#T(i+1)-1}(apuestas.s\cdot pagos.c)^{\#F(i+1)} \lor \\
    & res = recursos \cdot (apuestas.c\cdot pagos.c)^{\#T(i+1)}(apuestas.s\cdot pagos.c)^{\#F(i+1)-1})\\
P1A \lor P1B \equiv \\
    &(0\leq i \leq |eventos|)\land_L\\
    &(res = recursos \cdot (apuestas.c\cdot pagos.c)^{\#T(i)}(apuestas.s\cdot pagos.c)^{\#F(i)}\\
P1A \lor P1B \equiv & I
\end{split}
\end{equation}

Vimos que $\verb+wp(C;I)+ \to \{I \land B\}$, que es lo que queríamos probar.



\end{document}
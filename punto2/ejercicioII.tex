\documentclass[../document.tex]{subfiles}
\begin{document}

Tenemos el siguiente código, el cual llamaremos $C$
\begin{verbatim}
    res := recursos
    i := 0
    while (i < |eventos|) do
        if (eventos[i]) then
            res := res * apuesta.c * pago.c
        else
            res := res * apuesta.s * pago.s
        endif
        i := i+1
    endwhile
\end{verbatim}

Para demostrar que este código cumple con el requerimiento pedido, se deben comprobar dos cosas:

\begin{enumerate}
    \item la poscondición $Q$, de la tripla de Hoare $\{P\}C\{Q\}$, debe implicar el predicado que asegura la especificación.
    \item el predicado que requiere la especificación debe implicar a la precondición {P}.
\end{enumerate}

Para verificar esto, en primer lugar, debemos estudiar el ciclo que se encuentra dentro del código.

$\{P_c: i = 0 \land_{L} res = recursos \land_{L} apuesta_{c}+apuesta_{c}=1 \land_{L} pago_{c},pago_{s},apuesta_{c},apuestsa{s},recurso>0\}$
\begin{verbatim}
    while (i < |eventos|) do
        if (eventos[i]) then
            res := res * apuesta.c * pago.c
        else
            res := res * apuesta.s * pago.s
        endif
        i := i+1
    endwhile
\end{verbatim}
$\{Q_c: i = |eventos| \land_{L} res = recursos \cdot (apuestas.c\cdot pagos.c)^{\#T}(apuestas.s\cdot pagos.c)^{\#F}\}$

donde $\#T = \#(eventos, True)$ y $\#F = \#(eventos, F)$. Proponemos, a partir de esto, el siguiente invariante

$$I \equiv 0\leq i \leq |eventos| \land_L res = recursos \cdot (apuestas.c\cdot pagos.c)^{\#T(i)}(apuestas.s\cdot pagos.c)^{\#F(i)}$$

donde $\#T(i) = \#(eventos[0..i], True)$ y $\#F(i) = \#(eventos[0..i], F)$

Además, nos quedan la siguiente guarda y función variante

$$B \equiv i < |eventos|$$
$$f_v = |eventos|-1$$

La poscondición del ciclo, $Q_c$ es el $Q$ que estamos buscando. Por lo tanto, $Q_c$ debe implicar el predicado del asegura. La precondición del ciclo nos permitirá, luego, encontrar la precondición más débil del código, y así ver si la precondición implica ésta última.

Demostremos que $\{P_c\}S\{Q_c\}$ es una tripla de Hoare válida.

Digamos que \verb+C=C1;C2+, dónde \verb+C1+ es
\begin{verbatim}
    if (eventos[i]) then
        res := res * apuesta.c * pago.c
    else
        res := res * apuesta.s * pago.s
    endif
\end{verbatim}

y \verb+C2+ es
\begin{verbatim}
    i := i+1
\end{verbatim}

En principio, \textbf{sí el ciclo finaliza}, para ser válida deben ser ciertos los siguientes predicados
\begin{enumerate}
    \item $P_c \to I$
    \item $\{I \land B\}C\{I\}$
    \item $\{I \land \neg B\} \to Q_c$
    \item $\{I \land v_0 = fv\} S \{fv < v_0\}$
    \item $I \land fv<0 \to \neg B$
\end{enumerate}

\end{document}